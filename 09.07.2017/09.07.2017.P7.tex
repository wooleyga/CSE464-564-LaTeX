\documentclass{article}
\usepackage[margin=1in]{geometry}
\usepackage[english]{babel}
\usepackage{amsmath, amssymb, enumitem, color}
\linespread{1.5}
\parindent=0in

\begin{document}
(7) Do problem Sundstrom 4.2.17\\
\textbf{Evaluation of proofs}\\
See the instructions for Exercise (\textcolor{blue}{19}) on page \textcolor{blue}{100} from Section \textcolor{blue}{3.1}.
\begin{enumerate}[leftmargin=35px,label=(\textbf{\alph*})]
\item For each natural number $n$ with $n \geq 2, 2^n > 1 + n.$\\
\textit{\textbf{Proof.}} We let $k$ be a natural number and assume that $2^k > 1 + k.$ Multiplying both sides of this inequality by $2$, we see that $2^{k+1} > 2 + 2k.$ However, $2 + 2k > 2 + k$ and, hence,
\begin{equation*}
2^{k+1} > 1 + (k + 1)
\end{equation*}
By mathematical induction, we conclude that $2^n > 1 + n$.\hfill$\blacksquare$

\item Each natural number greater than or equal to 6 can be written as the sum of natural numbers, each of which is a 2 or a 5.

\textbf{\textit{Proof.}} We will use a proof by induction. For each natural number $n$, we let $P(n)$ be, "There exist non-negative integers $x$ and $y$ such that $n = 2x + 5y$." Since
\begin{align*}
    6 &= 3 \cdot 2 + 0 \cdot 5 & 7 &= 2 + 5\\
    8 &= 4 \cdot 2 + 0 \cdot 5 & 9 &= 2 \cdot 2 + 1 \cdot 5
\end{align*}
we see that $P(6)$, $P(7)$, $P(8)$, and $P(9)$ are true.\\
We now suppose that for some natural number $k$ with $k \geq 10$ that $P(6)$, $P(7), \dots P(k)$ are true. Now
\begin{equation*}
k + 1 = (k - 4) + 5
\end{equation*}

Since $k \geq 10$, we see that $k-4 \geq 6$ and, hence, $P(k - 4)$ is true. So $k - 4 = 2x + 5y$ and, hence, 
\begin{equation*}
k - 1 = (k - 4) + 5
\end{equation*}
Since $k \geq 10$, we see that $k - 4 \geq 6$ and, hence, $P(k - 4)$ is true. So $k - 4 = 2x + 5y$ and, hence, 
\begin{align*}
k+1 &= (2x + 5y) + 5\\
&= 2x + 5(y + 1)
\end{align*}

This proves that $P(k+1)$ is true, and hence, by the Second Principle of Mathematical Induction, we have proved that for each natural number $n$ with $n \geq 6$, there exist non-negative integers $x$ and $y$ such that $n = 2x + 5y$. \hfill$\blacksquare$\\
\end{enumerate}

(a) The proposition is correct but the proof is invalid as it does not prove the basis step of the Principle of Mathematical Induction. We will prove the proposition.\\
\textbf{Proposition.} For each natural number $n$ with $n \geq 2, 2^n > 1 + n.$\\
\textit{\textbf{Proof.}} We will prove the proposition by using the Principle of Mathematical Induction. For each $n \in \mathbb{N}$ such that $n \geq 2$, we let $P(n)$ be
\begin{equation*}
2^n > 1 + n.
\end{equation*}
We will first prove the basis step. That is, we prove the case when $n = 2.$ Note that $2^2 = 4 > 1 + 2 = 3.$ Since this inequality is true, we have proven $P(2)$ and consequently, the basis step.\\

We will next prove the inductive step. That is, we will prove that for each $k \in \mathbb{N}$ such that $k \geq 2$ if $P(k)$ is true, then $P(k + 1)$ is true. We assume that $P(k)$ is
\begin{equation}
2^k > 1 + k
\end{equation}
and that $P(k + 1)$ is
\begin{align}
2^{k + 1} &> 1 + (k + 1)\nonumber \\
&> 2 + k.
\end{align}
Multiplying both sides of $P(k)$ by 2, we obtain
\begin{align*}
2^k \cdot 2 &> 2(1 + k)\\
2^{k + 1} &> 2 + 2k
\end{align*}
Since $2 + 2k > 2 + k$, it follows that
\begin{equation*}
2^{k + 1} > 2 + 2k > 2 + k\\
\end{equation*}
Since $2^{k + 1} > 2 + k$ is the same as equation (\textcolor{blue}{2}), we have shown that if $P(k)$ is true, then $P(k + 1)$ is true. Hence, we have proved the inductive step and consequently, we have proved the proposition by the Principle of Mathematical Induction.\hfill$\blacksquare$ \\

(b) This is a true proposition with a correct proof, but the proof could be written better. We will prove the proposition.\\
\textbf{Proposition.} Each natural number greater than or equal to 6 can be written as the sum of natural numbers, each of which is a 2 or a 5.\\
\textit{\textbf{Proof.}} We will prove the proposition by the Principle of Mathematical Induction. For each $n \in \mathbb{N}$, we let $P(n)$ be, "There exist non-negative integers $x$ and $y$ such that $n = 2x + 5y.$" We begin with the basis step. Since
\begin{align*}
6 &= 3 \cdot 2 + 0 \cdot 5 & 7 &= 2 + 5\\
8 &= 4 \cdot 2 + 0 \cdot 5 & 9 &= 2 \cdot 2 + 1 \cdot 5
\end{align*}
we see that $P(6), P(7), P(8),$ and $P(9)$ are true.\\
We next prove the inductive step. We assume that for each $k \in \mathbb{N}$ such that $k \geq 10$, if $P(k)$ is true, then $P(k + 1)$ is true. We assume that
\begin{equation*}
k - 4 = 2x + 5y
\end{equation*}
Then,
\begin{align*}
k + 1 &= (k - 4) + 5\\
&= (2x + 5y) + 5\\
&= 2x + 5(y + 1)
\end{align*}
Since $y + 1$ is a non-negative integer, $P(k + 1)$ has been shown to be expressed in the form of $P(n)$. That is, we have proved that if $P(k)$ is true, then $P(k + 1)$ is true. Since we have proved the basis step and the inductive step, we have proved the proposition by the Principle of Mathematical Induction.\hfill$\blacksquare$
\end{document}