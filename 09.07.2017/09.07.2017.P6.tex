\documentclass{article}
\usepackage[margin=1in]{geometry}
\usepackage[english]{babel}
\usepackage{amsmath, amssymb, enumitem, color}
\linespread{1.5}
\parindent=0in

\begin{document}

(6) Do problem Sundstrom 4.2.13 (Use mathematical induction...!)

\textbf{13.} Prove or disprove each of the following propositions.

\begin{enumerate}[leftmargin=35px,label=(\textbf{\alph*})]
\item For each $n \in \mathbb{N}$, $\frac{1}{1 \cdot 2} + \frac{1}{2 \cdot 3} + ... + \frac{1}{n(n + 1)} = \frac{n}{n + 1}$.


\item For each natural number $n$ with $n \geq 3, \frac{1}{3 \cdot 4} + \frac{1}{4 \cdot 5} + ... + \frac{1}{n(n + 1)} = \frac{n - 2}{3n + 3}$.

\item For each $n \in \mathbb{N}$, 1$\cdot$2 + 2$\cdot$3 + 3$\cdot$4 + ... + $n(n + 1)$ = $\frac{n(n + 1)(n + 2)}{3}$.
\end{enumerate}

(a) We will prove the proposition.\\
\textbf{Proposition.} For each $n \in \mathbb{N}$, $\frac{1}{1 \cdot 2} + \frac{1}{2 \cdot 3} + ... + \frac{1}{n(n + 1)} = \frac{n}{n + 1}$.\\
\textit{\textbf{Proof.}} We will prove the proposition by using the Principle of Mathematical Induction. For each $n \in \mathbb{N}$, we let $P(n)$ be
\begin{equation*}
\frac{1}{1 \cdot 2} + \frac{1}{2 \cdot 3} + ... + \frac{1}{n(n + 1)} = \frac{n}{n + 1}.
\end{equation*}
We will first prove the basis step. That is, we will prove $P(1).$ Since 
\begin{equation*}
\frac{1}{1 \cdot 2} = \frac{1}{(1 + 1)},
\end{equation*} we have proven that $P(1)$ is true.\\

We will next prove the inductive step. That is, we prove that for each $k \in \mathbb{N}$ if $P(k)$ is true, then $P(k + 1)$ is true. We assume
\begin{align}
P(k) &= \frac{1}{1 \cdot 2} + \frac{1}{2 \cdot 3} + ... + \frac{1}{k(k + 1)} = \frac{k}{k + 1}\\
P(k + 1) &= \frac{1}{1 \cdot 2} + \frac{1}{2 \cdot 3} + ... + \frac{1}{k(k + 1)} + \frac{1}{(k + 1)((k + 1) + 1)} = \frac{k + 1}{(k + 1) + 1}
\end{align}
\\
Note that $P(k + 1) = P(k) + \frac{1}{(k + 1)((k + 1) + 1)}$. So to prove the inductive step, we will prove that by adding $\frac{1}{(k + 1)((k + 1) + 1)}$ to $P(k)$, it will equate to $P(k + 1)$. That is,
\begin{align*}
P(k) + \frac{1}{(k + 1)((k + 1) + 1)} &= \frac{k}{k + 1} + \frac{1}{(k + 1)(k + 2)}\\
&= \frac{k(k + 2) + 1}{(k + 1)(k + 2)}\\
&= \frac{k^2 + 2k + 1}{(k + 1)(k + 2)}\\
&= \frac{(k + 1)^2}{(k + 1)(k + 2)}\\
&= \frac{k + 1}{k + 2}\\
&= \frac{k + 1}{(k + 1) + 1}
\end{align*}

Since the algebra results to equation (\textcolor{blue}{1}), we have proved the inductive step by showing that if $P(k)$ is true, then $P(k + 1)$ is true. By proving the basis step and the inductive step, we have proven the proposition by the Principle of Mathematical Induction. \hfill$\blacksquare$ \\

(b) We will prove the proposition.\\
\textbf{Proposition.} For each natural number $n$ with $n \geq 3, \frac{1}{3 \cdot 4} + \frac{1}{4 \cdot 5} + ... + \frac{1}{n(n + 1)} = \frac{n - 2}{3n + 3}$.\\
\textit{\textbf{Proof.}} We will prove the proposition by using the Principle of Mathematical Induction. For each $n \in \mathbb{N}$ such that $n \geq 3$, we let $P(n)$ be
\begin{equation*}
\frac{1}{3 \cdot 4} + \frac{1}{4 \cdot 5} + ... + \frac{1}{n(n + 1)} = \frac{n - 2}{3n + 3}.
\end{equation*} \\

We begin the proof with the basis step. That is, we will prove $P(3)$. Letting $n = 3,$ we obtain
\begin{align*}
\frac{3 - 2}{3(3) + 3} &= \frac{1}{12}\\
&= \frac{1}{3 \cdot 4}
\end{align*}

This shows that 
\begin{equation*}
\frac{1}{3 \cdot 4} = \frac{3 - 2}{3(3) + 3}
\end{equation*}
which proves that $P(3)$ is true.\\

We will next prove the inductive step. That is, we prove that for each $k \in \mathbb{N}$ such that $k \geq 3$ if $P(k)$ is true, then $P(k + 1)$ is true. We assume
\begin{align}
P(k) &= \frac{1}{3 \cdot 4} + \frac{1}{4 \cdot 5} + ... + \frac{1}{k(k + 1)} = \frac{k - 2}{3k + 3}\\
P(k + 1) &= \frac{1}{3 \cdot 4} + \frac{1}{4 \cdot 5} + ... + \frac{1}{k(k + 1)} + \frac{1}{(k + 1)((k + 1) + 1)} = \frac{(k + 1) - 2}{3(k + 1) + 3}\\
&= \frac{k - 1}{3k + 6} = \frac{k - 1}{3(k + 2)} \nonumber
\end{align} 
\\Note that $P(k + 1) = P(k) + \frac{1}{(k + 1)((k + 1) + 1)}$. So to prove the inductive step, we will show that by adding $\frac{1}{(k + 1)((k + 1) + 1)}$ to $P(k)$, it will equate to $P(k + 1)$. That is, 
\begin{align*}
P(k) + \frac{1}{(k + 1)((k + 1) + 1)} &= \frac{k - 2}{3k + 3} + \frac{1}{(k + 1)((k + 1) + 1)}\\
&= \frac{k - 2}{3k + 3} + \frac{1}{(k + 1)(k + 2)}\\
&= \frac{(k - 2)(k + 2) + 3}{(3k + 3)(k + 2)}\\
&= \frac{k^2 - 4 + 3}{(3k + 3)(k + 2)}\\
&= \frac{k^2 - 1}{(3k + 3)(k + 2)}\\
&= \frac{(k + 1)(k - 1)}{3(k + 1)(k + 2)}\\
&= \frac{k - 1}{3(k + 2)}
\end{align*}

Comparing the result to equation (\textcolor{blue}{4}), we can see that the result has equated to $P(k + 1)$. Therefore, we have proved the inductive step by showing that if $P(k)$ is true, then $P(k + 1)$ is true. By proving the basis step and the inductive step, we have proven the proposition by the Principle of Mathematical Induction. \hfill$\blacksquare$ \\

(c) We will prove the proposition.\\
\textbf{Proposition.} For each $n \in \mathbb{N}$, $1\cdot2 + 2\cdot3 + 3\cdot4 + ... + n(n + 1) = \frac{n(n + 1)(n + 2)}{3}$.

\textit{\textbf{Proof.}} We will prove the proposition by using the Principle of Mathematical Induction. For each $n \in \mathbb{N}$, we let $P(n)$ be
\begin{equation*}
1\cdot2 + 2\cdot3 + 3\cdot4 + ... + n(n + 1) = \frac{n(n + 1)(n + 2)}{3}.
\end{equation*}
 We begin the proof with the basis step. That is, we will prove $P(1)$. Note that $1 \cdot 2$ is 2. Letting $n = 1$, we obtain
\begin{align*}
\frac{1 \cdot (1 + 1)(1 + 2)}{3} &= \frac{2 \cdot 3}{3}\\
&= 2.
\end{align*}

This shows that 
\begin{equation*}
1 \cdot 2 = \frac{1 \cdot (1 + 1)(1 + 2)}{3}
\end{equation*} and consequently, we have proved $P(1)$.\\

We will next prove the inductive step. That is, we will prove that for each $k \in \mathbb{N}$ if $P(k)$ is true, then $P(k + 1)$ is true. We assume that
\begin{align}
P(k) = 1\cdot2 + 2\cdot3 + 3\cdot4 + ... + k(k + 1) = \frac{k(k + 1)(k + 2)}{3}\\
P(k + 1) = 1\cdot2 + 2\cdot3 + 3\cdot4 + ... + k(k + 1) + (k + 1)((k + 1) + 1) = \frac{(k + 1)((k + 1) + 1)((k + 1) + 2)}{3}
\end{align}
Note that $P(k + 1) = P(k) + (k + 1)(k + 2)$. So to prove the inductive step, we will prove that by adding $(k + 1)(k + 2)$ to $P(k)$, it will equate to $P(k + 1)$. That is,
\begin{align*}
P(k) + (k + 1)(k + 2) &= \frac{k(k + 1)(k + 2)}{3} + (k + 1)(k + 2)\\
&= \frac{k(k + 1)(k + 2) + 3(k + 1)(k + 2)}{3}\\
&= \frac{(k + 1)(k + 2)(k + 3)}{3}\\
&= \frac{(k + 1)((k + 1) + 1)((k + 1) + 2)}{3}
\end{align*}
After algebra, the result is the same as equation (\textcolor{blue}{6}). Therefore, we have proved the inductive step by showing that if $P(k)$ is true, then $P(k + 1)$ is true. By proving the basis step and the inductive step, we have proven the proposition by the Principle of Mathematical Induction. \hfill$\blacksquare$

\end{document}