\documentclass{article}
\usepackage[margin=1in]{geometry}
\usepackage{amsmath, amssymb}
\linespread{1.3}
\parindent=0in

\begin{document}
(2) Explain the difference between a definition, a conjecture, a theorem, a proposition, and a lemma.\\

A \textbf{definition} is a logical mapping from a word or phrase to an object, description, or concept. It it unique from other terminologies because it makes no detail concerning truth or falsity.\\

A \textbf{conjecture} is a mathematical guess. It is a light consideration that does not have a formal proof by which it is declared with. Often, it will loosely be used to guess truth and falsity of a statement. \\

A \textbf{theorem} is a logically sound proposition (or series of logically sound propositions) that is often provided with a proof. Unlike conjectures and propositions, theorems are logically sound arguments of fact.\\

A \textbf{proposition} is a statement of fact that concerns truth and falsity. Propositions must have no ambiguity, unlike conjectures.\\

A \textbf{lemma} is a short theorem presented to give way to a larger theorem. Lemmas are theorems but they are unique in being short and given only in association with another theorem.
\end{document}

