\documentclass{article}
\usepackage[margin=1in]{geometry}
\usepackage[english]{babel}
\usepackage{amsmath, amssymb, amstext, graphicx}
\linespread{1.5}
\parindent=0in

\graphicspath{ {c:/users/gavin/documents/lightshot/} }

\begin{document}
\begin{center}
\includegraphics[scale=0.8]{screenshot_175}\\
\textbf{Figure 13.2}
\end{center}

\textbf{\textit{13.2-3}}\\
Let $a, b,$ and $c$ be arbitrary nodes in subtrees $\alpha, \beta,$ and $\gamma$, respectively, in the left tree of Figure 13.2. How do the depths of $a, b,$ and $c$ change when a left rotation is performed on node $x$ in the picture?\\

We assume that $\beta$ is the subtree starting at $x$.right. Then, we let $B$ = $x$.right $= \beta$.root. Since $b$ is an arbitrary node of $\beta$, which is rooted at $B$, we know that $B \neq$ \textsc{nil}. That is to say, $\beta$ is not an empty binary tree. We can make similar arguments to show that $\alpha$ and $\gamma$ are both not empty trees.\\

After a left rotation, since $x = x.$p.left $(= y.$left), we perform the following operations by \textsc{Left-Rotate}:
\begin{center}
$x$.right $= B.$left,\\
$y$.left $= B$,\\
$B$.left = $x$,
\end{center}
and likewise set the parents of the changed nodes appropriately. Then, our resulting tree looks roughly as so, where $B.$left denotes the left subtree which was originally rooted at $B$, and $B.$right denotes the right subtree which was originally rooted at $B$.
\begin{center}
\includegraphics[scale=0.8]{screenshot_175_leftrotate}
\end{center}
Hence, it is easy to see that $c$ does not change whatsoever in regards to depth as the depth of $\gamma$ remains unchanged. We can also easily see that the depth of $a$ is increased by one as the depth of $\alpha$ is increased by one. However, the depth of $b$ depends on its location. If $b$ lies in $B.$right, whose depth is decremented by one, then $b$ has a depth decremented by one. Likewise, if $b$ lies in $B.$left, whose depth is unchanged, then $b$ has an unchanged depth.

\end{document}